\chapter{Background}
\label{sec:back}

Human ears start to develop between fifth and seventh weeks of pregnancy. At this stage, the embryo face takes on more definition as mouth perforation, nostrils and ear indentations become visible. Forensic science literature reports that ear growth after the first four months of birth is highly linear [20]. The rate of stretching is five times greater than normal during the period from 4 months to the age of 8, after which, it is constant until the age of seventy when it again increases. Thus it can be said that ear remains almost unchanged during a substantial period of 62 years and, thus, it is one of the strong points of considering ear for biometric authentication.

Haar-based methods have given fairly better results for face detection as it is robust and fast. The different types of ear recognition systems include those of intensity-based, force-field based, 2D curves geometry, wavelet transformation, Gabor filters, SIFT, and 3D features. The force-field transforms gained popularity due to its uniqueness and efficiency [22]. Similar methods have also been implemented on other kinds of ear recognition systems [8][10].

Deep Methods have already come up and showing good performances on other face recognition systems which shows that it can also be applied to ear recognition systems. Hand-crafted feature detectors have not been able to work properly and are not robust, so deep features have been extracted to improve upon the performance. But one of the few drawbacks about deep learning is that it needs a large amount of data to train the model. There are not many ear databases that are too big but an attempt has been made to apply deep learning on a small scale database and analyze the results.
