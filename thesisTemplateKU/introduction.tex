\chapter{Introduction} \label{sec:intro} Embedded systems are becoming more
sophisticated and universal in our society, and are increasingly relying on
Real-Time Operating Systems (RTOS) to deliver precise, predictable, and robust
behavior.  Two fundamental challenges in the design of RTOS kernels are the
minimization of system overhead and jitter \cite{stankovicJitter}.  With a
shared computational resource such as a CPU, the execution of system services
themselves takes away from much needed application processing time.  By
minimizing the system overhead, more computational cycles are available to the
application.  Minimizing the jitter in the system allows for more precise,
deterministic scheduling of threads which allows the system to respond to
overhead in our system.

The rest of the paper is organized as follows.  The problem statement of this
thesis work is developed in chapter \ref{sec:problem}.  The subject of improved
precision and predictability of scheduling within an RTOS are discussed in
chapter \ref{sec:back}.  Related works in the subject area are discussed in
chapter \ref{sec:related}.  Chapter \ref{sec:design} describes the iterative
redesigns of the scheduler module over the course of my thesis work.  The
initial design of the scheduler module \cite{Agron:rt} is discussed in section
\ref{sec:build1}.  The second redesign of the scheduler module is covered in
section \ref{sec:build2}, which includes the work done to further reduce
scheduling overhead and jitter.  Section \ref{sec:build3} covers the final
scheduler module redesign which extends thread management and scheduling
services to both hardware and software-based threads in the HybridThreads
system.  Performance results of the various versions of the scheduler module,
both as an independent entity and actual integrated performance measurements
are shown in chapter \ref{sec:results}.  The evaluation of the design and
implementation of the scheduler module, and comments on future work to be done
on the scheduler module and other components within HybridThreads, a
multithreaded RTOS kernel
\cite{andrews04a,andrews04b,niehaus03,etfa05,Peck:dq}, are discussed in chapter
\ref{sec:conc}.

