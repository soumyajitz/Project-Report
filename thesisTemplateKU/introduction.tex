\chapter{Introduction} \label{sec:intro} Biometric authentication of people based on various anatomical characteristics, like eye, ear, face, iris, and fingerprint have attracted lots of attention during the past few decades, and some of these techniques have already been successfully applied for recognition and authentication. However, many systems are not very robust and may fail to work under certain conditions. Biometric ear recognition is a relatively new technique that may surpass the existing systems due to several significant reasons. For ex- ample, the acquisition of ear images is relatively easy and, unlike iris, can be captured without the co-operation of individuals 


The rest of the paper is organized as follows.  The problem statement of this
thesis work is developed in chapter \ref{sec:problem}.  The subject of improved
precision and predictability of scheduling within an RTOS are discussed in
chapter \ref{sec:back}.  Related works in the subject area are discussed in
chapter \ref{sec:related}.  Chapter \ref{sec:design} describes the iterative
redesigns of the scheduler module over the course of my thesis work.  The
initial design of the scheduler module \cite{Agron:rt} is discussed in section
\ref{sec:build1}.  The second redesign of the scheduler module is covered in
section \ref{sec:build2}, which includes the work done to further reduce
scheduling overhead and jitter.  Section \ref{sec:build3} covers the final
scheduler module redesign which extends thread management and scheduling
services to both hardware and software-based threads in the HybridThreads
system.  Performance results of the various versions of the scheduler module,
both as an independent entity and actual integrated performance measurements
are shown in chapter \ref{sec:results}.  The evaluation of the design and
implementation of the scheduler module, and comments on future work to be done
on the scheduler module and other components within HybridThreads, a
multithreaded RTOS kernel
\cite{andrews04a,andrews04b,niehaus03,etfa05,Peck:dq}, are discussed in chapter
\ref{sec:conc}.

