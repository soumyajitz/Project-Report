\chapter{Introduction} \label{sec:intro} Biometric authentication of people based on various anatomical characteristics, like eye, ear, face, iris, and fingerprint have attracted lots of attention during the past few decades, and some of these techniques have already been successfully applied for recognition and authentication. However, many systems are not very robust and may fail to work under certain conditions. Biometric ear recognition is a relatively new technique that may surpass the existing systems due to several significant reasons. For example, the acquisition of ear images is relatively easy and, unlike iris, can be captured without the co-operation of individuals \cite{pflug2012ear}

Human ear contains rich and stable features which are more reliable than face fea- tures, as the structure of the ear is not subject to change with age. It has also been found out that no two ears are exactly the same even for identical twins \cite{abaza}. The detailed structure of ear is not only very unique but also permanent, since the shape of a human ear never shows drastic changes over the course of life. The research on ear identification was first conducted by Bertillon, a French criminologist, in 1890. The process was refined by American police officer, Iannarelli [20], who divided the ear based on vari- ous distinctive features of seven parts: i.e. helix, concha, antihelix, crux of helix, inter- tragic notch, tragus, and antitragus [3].

Here, we propose to use two scale and rotation invariant feature detectors, i.e. SIFT (scale invariant feature transform) and SURF (speed up robust features), for ear recognition. Both SIFT and SURF extract specific interest points from an image and generate descriptors for the feature points to a form a reliable matching results.\\
%Path in Mac Format
\begin{figure}[t]
	\DeclareGraphicsExtensions{.pdf,.png,.jpg}
	\includegraphics[width=\textwidth]{Figures/Figure1}
	\caption{The pipeline of the proposed Ear Recognition System}
	\label{fig:Figure1}
\end{figure}
Extensive experiments have been carried out on two different sets of databases to evaluate their performance with respect to various rotations and scales. One of the most important feature of ear images is its easiness in acquisition, however, the acquired images may be in different scales, rotations, and illumination. The scale and rotation invariant property of the SIFT and SURF algorithms makes them perfect for ear authentication under various circumstances.

 A new concept in the field of machine learning and computer vision has come up which has surpassed the traditional object recognition methods. This new approach is called deep learning. Deep Learning is a branch of Machine Learning which has multiple levels of representations and abstractions. It is basically a rebranding of the term Artificial Neural Networks. Deep Learning algorithms have already been applied in Apple's Siri, Google's Streetview etc. 

The rest  is organized as follows. Some background and related research are discussed in Section 2; the proposed method is presented in details in Section 3; some experimental results and analysis are given in Section 4; and the paper is concluded in Section 5.
